%Show model performances on validation data set, and specify which was the selected model.
\subsection{Caps per hour}
In table \ref{table:MSE.caps}, one can see the mean squared errors when testing the corresponding model on a validation set. Obviously, the linear model performed best.
\begin{table}
\caption{MSEs for Caps/h models}
\label{table:MSE.caps}
\centering
\begin{tabular}{l|l}
\hline
\textbf{Model}&\textbf{MSE}\\
\hline
Linear&4.224231\\
Generalized linear model&14856.01\\
Quadratic generalized linear model&14855.91\\
\hline
\end{tabular}
\end{table}


\subsection{Survival time}
Since we wanted to penalize the models deviating by a high percentage\footnote{It is worse being $\pm$1 hour when survival time is 2 hours than 10 hours.}, we used the mean squared logarithmic error\footnote{I.e. $mean((log(x)-log(y))^2)$.} when comparing models. This is a problem since we have not used this error function when training the models.\\

Table \ref{table:MSLE.survival} contains the error data for the models for one of the runs. The data is sorted by increasing error.\\

When looking at the summaries for the different polynomial models, we see signs of overfitting since the higher order terms have unsignificant p-values. Hence, we select the fully linear model.
\begin{table}
\caption{Example MSLEs for survival time models}
\label{table:MSLE.survival}
\centering
\begin{tabular}{l l|l}
\hline
&\textbf{Model}&\textbf{MSLE}\\\hline
1 & Linear & 0.03743585 \\
2 & L-squared & 0.03757349 \\
3 & E-cubed & 0.03802546 \\
4 & E-squared & 0.03806007 \\
5 & Level-squared & 0.03990369 \\
6 & Level-cubed & 0.04001568 \\
7 & L-cubed & 0.04413953 \\
8 & Random forest & 0.06508721 \\
\hline
\end{tabular}
\end{table}